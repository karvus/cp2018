\documentclass[a4paper,11pt]{article}
\usepackage[utf8]{inputenc}
\usepackage[T1]{fontenc}
\usepackage[scaled=0.89]{inconsolata}
\author{Thomas Stenhaug \texttt{<tsten16@student.sdu.dk>}}
\usepackage[a4paper,top=2cm,bottom=2cm,left=4cm,right=4cm]{geometry}
\usepackage[iso,danish]{isodate}
\usepackage{fancyvrb}
%\usepackage{fancyhdr}
% \pagestyle{fancy}
% \fancyhf{}
% \lhead{T. Stenhaug}
% \rhead{DM519 exam \quad \quad \quad \today}
% \chead{}
% \lfoot{}
% \cfoot{\thepage}
% \rfoot{}
%\renewcommand{\footrulewidth}{0pt} %
%\setcounter{secnumdepth}{2}
%\setcounter{tocdepth}{2}
\date{\today}
\title{DM519 Spring 2018 Exam}
\begin{document}

\maketitle

\setlength{\baselineskip}{1.44\baselineskip}

\section{Methodology}

The central units of work in the project, are files containing a
plaintext line of comma separated integer-values.  I have chosen to
make a wrapper-class \texttt{NumberFile} around Java's \texttt{Path}s to
facilitate working with these files.

A trait shared between my solutions to \texttt{m1, m2} and
\texttt{m3}, is that they instantiate a
\texttt{BlockingDeque<NumberFile>} as the main shared datastructure
for production and consumption.  Initially, an executor for consumers,
with one consumer for each processor, is created.  A producer then
runs from the main thread, producing to the
\texttt{BlockingDeque<NumberFile>}; in the main thread, because the
operation is I/O-bound, so it didn't seem worth it to invest
complexity in distributing the work over threads.  (Although it uses
\texttt{parallel()} in a stream-chain, just in case the JRE knows how
to optimize it.)

While the main thread is producing work-units, the consumers, one for
each processor, accepts and perform their work.

\subsection{Synchronizing when consumers are done}

For \texttt{m1} and \texttt{m3}, the mechanism for synchronizing the
consumers is a \textit{poison pill} that is fed to the queue of
\texttt{NumberFile}s.  The consumers gracefully exit their loop after
encountering the poison pill, so after finishing producing work, the
main thread shuts down the executor, and \texttt{await}s it, which
will block until consumers are all done.  In the case of \texttt{m1},
that concludes its job.  \texttt{m3} instead, needs to do some
additional computations on the collected data.  These computations are
run in parallel, governed by a separate executor.  Here, the
synchronization is implemented by waiting for said computations as
futures, while constructing the final result.

\texttt{m2} is different, in that instead of the result being an
aggregate of all files, it is an arbitrary element from the
\texttt{BlockingDeque<NumberFile>}, matching some criteria.  This
element is represented by a \texttt{CompletableFuture}, which is
waited for in the main thread after, production of
\texttt{NumberFiles} is completed
\footnote{Please note that
  production of numberfiles will also shortcut, when the
  \texttt{CompletableFuture} is done.  This is achieved by a pun on
  the stream operator \texttt{allMatch}
}, and is completed by
the first consumers that find an element matching the criteria.
Consumers exit either when they discover that the
\texttt{CompletableFuture} is completed, or they are cancelled as a
result of their executor being shut down.

\subsection{Post processing}

While \texttt{m1} and \texttt{m2} are done with their work when their
consumers terminate, \texttt{m3} requires post-processing.  Its
consumers are also producers, adding work to a handful of concurrent
datastructures.   

\section{Advantages}

\subsection{Choice of datastructures, performance}

I chose to use ``off-the-shelf'' concurrent datastructures for this,
instead of writing my own.  In \texttt{m3}, for example, I keep a
\texttt{ConcurrentMap<Integer, LongAdder>} which keeps a tally of
number → occurences.  Consumer threads are all writing to this
structure, so it is in high contention.  I update it like so
:

\begin{verbatim}
frequencies.computeIfAbsent(i, k -> new LongAdder()).increment();
\end{verbatim}

This results in lock-free, thread-safe operation on this
high-contention data.

One of the return values from \texttt{m3} is an ordered
\texttt{List<Path>}.  Initially I thought to populate a
\texttt{ConcurrentLinkedQueue} in the consumers, and obtain the
ordering during the post-processing.  This would be an \(O(n \lg n) \)
operation, in a single thread.  Instead I chose a
\texttt{ConcurrentSkipListSet}, which maintains $O(\lg n)$ for each
$n$ insertions, but with ``for free'' parallelization, since the
insertions are spread across the already parallelized consumers.

\section{Limitations}

Every line in the files are read into memory in their entirety.  For
small inputs, this should be fine.  In scenarios where number of
processors grow, RAM per thread shrinks and line size grows, it
becomes a problem.

There are many occurrences of streams-usage, which would be likely be
faster if written out explicitly as loops.

Method-signatures are somewhat noisy and possibly hard to read,
especially for \texttt{StatsComputer}.  This could have been improved
by introducing abstractions over the arguments.

There is no rhyme nor rythm to exception-handling; instead, exceptions
are mostly silently ignored.

No tests are supplied, so changing the program requires
experimentation and eyeballing to assess its correctness.

\texttt{m1} in particular has a worse run-time than that of a
synchronous reference implementation.

\end{document}